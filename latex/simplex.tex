%This is a very simple LaTeX article that illustrates
%some of the features of LaTeX.
%The percent sign is comment.
%There are two ways to make this file into a PDF.  At
%the Linux prompt, do the following:
%
% latex simplex.tex
% latex simplex.tex
% dvips -o simplex.ps simplex.dvi
% ps2pdf simplex.ps
%
%Alternatively, you could use:
%
% pdflatex simplex.tex
%
%The latter works well *only* if you do not have EPS graphics
%embedded in your file.
%%%%%%%%%%%%%%%%%%%%%%%%%%%%%%%%%%%%%%%%%%%%%%%%%%
\documentclass[12pt,onecolumn]{article} %Set the overall font size and layout here

\title{A Simple \LaTeX\ Article}
\author{Amalie K.\ Student} %The "\ " makes a single space after the perioad
\date{\today}

\begin{document}

\maketitle %Create the title (title, author, date) that we made above

\abstract{This is a very simple \LaTeX\ article that illustrates some of the
features of \LaTeX.}  This abstract is formatted by a default command.

\section{Introduction to Equations}
It's quite easy to create section headings.  We can format them later.  Once you learn the syntax, equations are a snap!  You could typeset simple equations inline to show that $\vec{F}=m\vec{a}$, or you could separate the equations from text to show that
$$V=-\frac{fR}{2}\pm\left(\frac{f^2R^2}{4}-R\frac{\partial\Phi}{\partial n}\right)^{1/2}.$$
Notice the punctuation (i.e., the ``.'') that often accompanies equations. The \texttt{$\backslash$displaymath} command does the same thing as the \texttt{\$\$}.
In scientific papers, it's often useful to number the equations, such as
\begin{equation}
\label{thermal}
\frac{\partial\vec{V_g}}{\partial \ln p}=-\frac{R}{f}\vec{k}\times\nabla_p T.
\end{equation}
Then we can refer to the thermal wind relation (Eq.~\ref{thermal}) using a reference label instead of by a specific number.  That way, both the equation number and reference will change automatically if we add more numbered equations to the document! 

A blank line tells \LaTeX\ to make a new paragraph.

\section{Spacing and Formatting}
\LaTeX\ is extremely flexible with spacing \hspace{5mm}
and text formatting, which is something that Microsoft Word can't quite keep up with.\\ %make a new line with "\\"
\vspace{1in} %Make a vertical space; you can also move up!
\begin{center}
\Huge{Sometimes,}
\huge{you}
\LARGE{may}
\Large{want}
\large{to}
\normalsize{change}
\small{the}
\footnotesize{font}
\scriptsize{size}
\tiny{and appearance}
\normalsize
\textbf{using font}
\textsf{size}
\textit{and}
\textsc{style}
\textsl{commands}.
\end{center}

\vspace{10mm}
\raisebox{0pt}[0pt][0pt]{\Large%
\textbf{Play\raisebox{-0.3ex}{wi}%
\raisebox{-0.7ex}{th}%
\raisebox{-1.2ex}{t}%
\raisebox{-2.2ex}{e}%
\raisebox{-4.5ex}{x}%
\raisebox{-6.5ex}{t}}}

\begin{enumerate}
\item apple
\item orange
\item strawberry
\end{enumerate}

\end{document}

